%Das Dokument ist zweizeilig!
\documentclass[11pt,a4paper,twocolumn]{scrartcl}
\usepackage[a4paper, left=2cm, right=2cm, %top=3.5cm, 
bottom=0cm]{geometry}
\usepackage[utf8]{inputenc} %deutsche Umlaute
\usepackage[ngerman]{babel} %deutsche Sprache
\usepackage{graphicx} %für Bilder
\usepackage{booktabs} %für \toprule \midrule \bottomrule in Tabellen
\usepackage{siunitx}
\usepackage{lmodern}
\usepackage[export]{adjustbox}
\usepackage[inline]{enumitem}   
\usepackage{color,colortbl}
\usepackage{rotating}
\usepackage{wrapfig}
\usepackage{colortbl}
\usepackage[table,dvipsnames]{xcolor}
\usepackage{amsmath}
\usepackage{float}
\usepackage{multicol}
\usepackage{csquotes}
\usepackage[gen]{eurosym}
\usepackage{amsmath}
\usepackage{esint}
\usepackage{xcolor}
\usepackage{multicol, supertabular}
\usepackage{amsmath}
\usepackage{caption}
\usepackage{geometry}
\usepackage{longtable}
\usepackage{MnSymbol}
%\usepackage{hyperref}
\newcommand{\diff}{\mathop{}\!\mathrm{d}}
\usepackage[onehalfspacing]{setspace}
\usepackage{multirow}
\usepackage{tabularx}
\setlength{\parindent}{0pt}
\sisetup{output-decimal-marker = {,}}
%Eingabe Kopfzeile
\newcommand{\Unterichtsreihe}{Formelsammlung} 
\newcommand{\Thema}{Maschinenelemente: Schweißverbindungen} 
\newcommand{\Ersteller}{Klasse: \dotfill \qquad \qquad} 
\newcommand{\Semester}{Datum: \dotfill}  
%Breite der Spalten
\newcommand{\ceins}{2.5cm}
\newcommand{\czwei}{5cm}  
\newcommand{\cges}{7cm}    

\usepackage[headsepline,automark]{scrlayer-scrpage} 
%Ausgabe Kopfzeile
\clearpairofpagestyles
\chead{%
	\begin{tabular}{cp{8cm}}
	\multirow{3}{2.7cm}{\includegraphics[width=2.7cm,margin=0pt 0ex 0pt -4.5pt]{logo}}&\multicolumn{1}{l}{Schule}\\
	&\Unterichtsreihe\\
	&\Thema
\end{tabular}%
}%
\addtokomafont{pagehead}{\normalfont}
%\cfoot{\pagemark}
\pagestyle{headings}
\definecolor{bluee}{RGB}{100,149,237}


%Anfang Dokument
\begin{document}
\renewcommand{\arraystretch}{1.1}	
\begin{tabular}{p{\cges}}
	\arrayrulecolor{blue}
	\begin{tabular}{|p{\ceins}|p{\czwei}|}
	\multicolumn{2}{c}{\textbf{\Large Allgemein}}\\\hline
	Zulässige Normalspannung&$\sigma_{\text{zul}}=\frac{R_e}{S_M}=\frac{R_e}{1,1}$\\\hline
	Zulässige Schubspannung&$\tau_{\text{zul}}=\frac{R_e}{\sqrt{3}\cdot S_M}=\frac{R_e}{\sqrt{3}\cdot 1,1}$\\\hline
\end{tabular}
%Tabellenfarbe
\arrayrulecolor{red}
\begin{tabular}{|p{\ceins}|p{\czwei}|}
	\multicolumn{2}{c}{\textbf{\Large Zugstäbe}}\\\hline
	Schlankheit&$$\frac{b}{t}\leq\left(\frac{b}{t}\right)_{\text{Grenz}}$$\\\hline
	Zugspannung&$$\sigma_z=\frac{F_N}{A}\leq\sigma_{\text{zul}}$$\\\hline
	Biegemoment&$$ M_b=F\cdot (e+0,5t)$$\\\hline
	Biegespannung&$$ \sigma_b=\frac{M_b}{W}=\frac{M_b\cdot e}{I}$$\\\hline
	\multicolumn{2}{|c|}{\includegraphics[width=\cges]{logo}}\\\hline
	Maximale Spannung&$$ \sigma_{\text{max}}=\sigma_z+\sigma_b\leq \sigma_{\text{zul}}$$\\\hline
\end{tabular}
	\arrayrulecolor{green}
\begin{tabular}{|p{\ceins}|p{\czwei}|}
	\multicolumn{2}{c}{\textbf{\Large Druckstäbe}}\\\hline
	Druckspannung&$$\sigma_d=\frac{F_N}{A}\leq\sigma_{\text{zul}}$$\\\hline
	Grobe Vorbemessung&\begin{gather*}A_{\text{erf}}\approx\frac{F}{12}...\frac{F}{10}\\ I_{\text{erf}}\approx 0,12\cdot F\cdot l_k^2\end{gather*} mit $F$ in kN, $l_k$ in m, $A_{\text{erf}}$ in \SI{}{cm^2} und $I_{\text{erf}}$ in \SI{}{cm^4}	\\\hline
\end{tabular}
%Hier wird der Umbruch gemacht.
\end{tabular}
\begin{tabular}{p{\cges}}
	\arrayrulecolor{green}
	\begin{tabular}{|p{\ceins}|p{\czwei}|}	\hline
		Schlankheits-grad&\begin{gather*}\lambda_{kx}=\frac{l_k}{\sqrt{\frac{I_x}{A}}}\\\lambda_{ky}=\frac{l_k}{\sqrt{\frac{I_y}{A}}}\end{gather*}\\\hline
	\multicolumn{2}{|c|}{\includegraphics[width=\cges]{logo}}\\\hline
	Ideale Knicklast&$$\lambda_{a}=\pi\sqrt{\frac{E}{R_e}}$$\\\hline
	bezogener Schlankheitsgrad&\begin{gather*}\overline{\lambda_{kx}}=\frac{\lambda_{kx}}{\lambda_{a}}\\\overline{\lambda_{ky}}=\frac{\lambda_{ky}}{\lambda_{a}}\end{gather*}\\\hline
		Druckkraft in vollplastischen Zustand&$$F_{pl}=\frac{A\cdot R_e}{S_M}$$\\\hline
		Bemessungswert der Stab-Druckkraft&$$\frac{F}{\kappa\cdot F_{pl}}\leq1$$\\\hline
\end{tabular}	
	

\end{tabular}

\begin{tabular}{p{\cges}}
		\arrayrulecolor{gray}
	\begin{tabular}{|p{\ceins}|p{\czwei}|}
		\multicolumn{2}{c}{\textbf{\Large Knotenbleche}}\\\hline
		Normal-spannung&$$\sigma=\frac{F}{b\cdot t_K}\leq\sigma_{\text{zul}}$$\\\hline
		\multicolumn{2}{|c|}{\includegraphics[width=\cges]{logo}}\\\hline
	\end{tabular}
	\arrayrulecolor{Tan}
	\begin{tabular}{|p{\ceins}|p{\czwei}|}
		\multicolumn{2}{c}{\textbf{\Large Einfache Biegeträger}}\\\hline
		Normal-spannung&$$\sigma=\frac{F_N}{A}+\frac{M_x}{W_x} \leq \sigma_{\text{zul}}$$\\\hline
		Schubspannung&$$\tau_m=\frac{F_q}{A_s}\leq \tau_{\text{zul}}$$\\\hline
		Vergleichs-spannung&\begin{gather*}\sigma_v=\sqrt{\sigma^2+3\tau_m^2}\leq\sigma_{\text{zul}}\end{gather*}\\\hline
	\end{tabular}
	\arrayrulecolor{ForestGreen}
\begin{tabular}{|p{\ceins}|p{\czwei}|}
	\multicolumn{2}{c}{\textbf{\Large Schweißnähte im Stahlbau}}\\\hline
	Stumpfnähte&$$ a=t_{\text{min}}$$\\\hline
	Kehlnähte&\begin{gather*}\SI{2}{mm}\leq a \leq 0,7t_{\text{min}}\\ a\leq\sqrt{t_{\text{max}}}-\SI{0.5}{mm}\end{gather*}\\\hline	
\end{tabular}
\end{tabular}
\begin{tabular}{p{\cges}}
	\arrayrulecolor{ForestGreen}
\begin{tabular}{|p{\ceins}|p{\czwei}|}\hline
	\multicolumn{2}{c}{\quad}\\
	\multicolumn{2}{c}{\textbf{\Large Festigkeitsnachweis}}\\\hline
	Vergleichs-spannung&\begin{gather*}\sigma_{\text{wv}}=\sqrt{\sigma_{\bot}^2+\tau_{\parallel}^2+\tau_{\bot}^2}\leq\sigma_{\text{w zul}}\end{gather*}\\\hline	
	\multicolumn{2}{|c|}{\includegraphics[width=\cges]{logo}}\\\hline
	Beanspruchung auf Zug, Druck oder Schub&\begin{gather*}\left. \begin{matrix}\sigma_{\bot} \\\tau_{\bot} \\	\tau_{\parallel} \end{matrix}\right\}=\frac{F}{A_{\text{w}}}=\frac{F}{\Sigma(a\cdot l)}\leq \\\sigma_{\text{w zul}}=\tau_{\text{w zul}} \end{gather*}\\\hline
	Auf Biegung und Querkraft beanspruchter Kehlnahtanschluss&\begin{gather*}\sigma_{\bot}=\frac{M}{I_{\text{w}}}\cdot y\leq \sigma_{\text{w zul}}\\\tau_{\parallel}=\frac{F_\text{q}}{A_{\text{wS}}}\leq\tau_{\text{w zul}}\\=\sigma_{\text{w zul}} \end{gather*} ($A_{\text{wS}}$ ist die Querschnittsfläche die parallel zur Querkraft verläuft, z.B. $A_\text{w3}$)\\\hline
	Flächen-trägheits-moment der Schweißnaht&$$I_{\text{w}i}=\frac{b\cdot h^3}{12}$$\\\hline
	\multicolumn{2}{|c|}{\includegraphics[width=\cges]{logo}}\\\hline
	
	\end{tabular}

\end{tabular}

\begin{tabular}{p{\cges}}
	\arrayrulecolor{ForestGreen}
	\begin{tabular}{|p{\ceins}|p{\czwei}|}\hline
	Gesamtes Flächen-trägheits-moment&\begin{gather*}I_{\text{w}}=\Sigma I_{\text{w}i}+\Sigma y_i^2\cdot A_{\text{w}i}  \end{gather*}\\\hline
Bei Anschlüssen mit doppeltsymmetrischen I-Profilen &\begin{gather*}\sigma_{\bot}=\sigma_{\bot\text{zd}}+\sigma_{\bot\text{b}}\\=\frac{\frac{F_N}{2}+\frac{M}{h_F}}{A_{\text{wF}}} \leq \sigma_{\text{w zul}} \end{gather*}\\\hline
\multicolumn{2}{|c|}{\includegraphics[width=\cges]{logo}}\\\hline
	\end{tabular}
\arrayrulecolor{black}	
	\begin{tabular}{|p{\ceins}|p{\czwei}|}\hline
	\multicolumn{2}{c}{\textbf{\Large Legende}}\\\hline
	\textbf{Formel-zeichen}&\textbf{Bedeutung}\\\hline 
	$a$&Schweißnahthöhe\\\hline
	$A_\text{w}$&Querschnittsfläche der Schweißnaht\\\hline
	$b$&Breite\\\hline
	$F_N$&Normalkraft\\\hline
	$F_q$&Querkraft\\\hline
	$t$&Dicke\\\hline
	$t_K$&Dicke des Knotenblechs\\\hline
	$M_b$&Biegemoment\\\hline
	$I$&Flächenträgheitsmoment\\\hline
	$I_{\text{w}}$&Flächenträgheitsmoment der Schweißnaht\\\hline
	$R_e$&Streckgrenze\\\hline
	$S_M$&Sicherheitsfaktor ($S_M=1,1$)\\\hline
\end{tabular}
\end{tabular}


\end{document}